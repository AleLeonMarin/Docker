\documentclass{article}
\usepackage{listings}

\title{Docker Documentation}
\author{Alejandro Leon Marin}

\begin{document}

\maketitle
\newpage
\begin{center}
    \section*{Introduction}
        In this document we will cover the basics of Docker, a containerization platform 
        that allows you to run applications in a controlled environment. 
        Also it will help us to understand the basic concepts of Docker and how to use it. 
        We'll cover the definition of containers, images, and the basic commands to 
        run docker containers.

\end{center}

\newpage
\section*{What is Docker?}

Docker is a versatile platform designed for developing, packaging, and running applications within a controlled environment. 
This platform utilizes containerization, allowing applications to run efficiently on various systems while minimizing resource usage. 
Developers widely adopt Docker due to its ability to streamline the installation and execution of applications.

A container, in this context, serves as a packaged unit comprising an application and its dependencies, such as libraries and system tools. 
It functions as an isolated environment, ensuring that the application runs consistently across different systems with Docker installed. 
Consider a container as a virtual box containing your application's code, language-specific components (e.g., HTML and Node.js), 
and even environment variables within a designated directory like ".env."

Containers can be easily transported between systems, facilitating seamless sharing among developers and operations teams. 
Docker containers are stored in repositories, which are akin to GitHub but specifically for containers. There are two types of repositories:

1. Public repositories: Open to the public, allowing anyone to view the containers they contain, with Docker Hub being a prominent example.
2. Private repositories: Access to these repositories is restricted, and only individuals with authorized access can view the containers stored 
within.

Docker Hub provides a wide array of containers for databases, web servers, and other tools that developers can readily incorporate into their 
projects. When you run a container, you are essentially launching an instance of an image. This approach becomes particularly advantageous in 
collaborative projects, where team members may have different software versions installed. Docker simplifies this by enabling users to run an 
image, based on a specific Linux distribution, with the necessary dependencies—eliminating the need to install them individually.

In summary, Docker facilitates a more efficient and standardized development and deployment process by encapsulating applications and 
their dependencies into containers, which can be easily shared and executed across diverse environments.

\section*{What is an image?}
So an image is a file that contains all the necessary files to run a container. It includes the code, the runtime, 
the libraries, the environment variables, and the configuration files and the container makes that all of this files run in a controlled 
environment. So the image is the file that you can share with other developers and with the operations team. So what's a container?


\section*{What is a container?}

A container are layers trough layers of images. Where the lowest layer is most of the time a linux system distribution. The most used is called Alpine
linux, this is because it's a very small distribution and it's very fast to download. So over that layer we will have many layers of images until we get 
to the layer of our application. So the magic of the containers are that they are very small talking in terms memory space if we compare them 
with actual virtual machines where the vm's can take up Gigabytes of memory and the container can take MegaBytes of memory. And why virtualization? 

\section*{Why virtualization?}
Docker is a way of virtualization. So let's talk about virtualization. Let's take VM as Virtual Machine. 
So VM is based on 3 layers 
\begin{itemize}
    \item The hardware: It is where the VM is running.
    \item The kernel: It is the layer that is between the hardware and the VM.It's the layer that allows the VM to run on the hardware.
    \item The apps: Not much to explain, this are the applications we use when we work with the VM.
\end{itemize}
So when we talk about VM we virtualise The apps and the kernel. In this case the kernel could be a linux distribution, windows iso or mac os iso. 
And this makes that the images of the VM's size up to Gigabytes. In docker case we only virtualize the apps. So with the kernel docker uses
the host kernel. So if you are executing a container in a linux host, the container will use the linux kernel of the host. 
Same happen with windows and MacOs. 


\section*{Basic commands}
Let's start with the basic commands of docker. First to run the commands you need to run docker desktop running. Also to check if 
you have docker installed you can run the command
\begin{lstlisting}
    docker --version
\end{lstlisting}

Also docker recommend to run the hello-world image to check if the installation is correct. To do that you can run the command
\begin{lstlisting}{language=bash}
    docker run hello-world
\end{lstlisting}
 
This command will download the hello-world image and run it. If everything is correct you will see a message that says that the installation 
is correct.

Now our first command will be to list all the images that we have in our local machine. To do that we can run the command
\begin{lstlisting}{language=bash}
    docker images
\end{lstlisting}

Now to download our image we can run the command
\begin{lstlisting}{language=bash}
    docker pull <image-name>
\end{lstlisting}

In this case <image-name> could python for example. This will download the python image from the docker hub or Node, or mysql, or Postgres, etc.
Another thing to have in consideration is that you can pull a image with a specific version. For example if you want to pull the python 3.8.5 
you can run the command
\begin{lstlisting}{language=bash}
    docker pull python:3.8.5
\end{lstlisting}

But if you use the command docker pull python you will download the latest version of python.

And in the process of the pull you will see graphically how the layers of the image are being downloaded and this will help to understand that concepts.



\section*{Conclusion}



\end{document}